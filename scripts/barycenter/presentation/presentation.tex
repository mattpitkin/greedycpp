\input{mmd-beamer-header-11pt}
\def\mytitle{Reduced Order Modelling for Solar System Barycentring}
\def\mydate{10 July 2017}
\def\myauthor{Matthew Pitkin}
\def\affiliation{University of Glasgow}
\def\latexxslt{beamer}
\def\latexmode{beamer}
\def\theme{m}
\def\event{LVC CW group}
\input{mmd-beamer-begin-doc}


% A presentation to the LVC CW group on using reduced order modelling to speed up solar
% system barycentring time delay calculations
%
% Note: comments can be included in the LaTeX file by surrounding them with html style comment
% blocks and a % sign


\begin{frame}

\frametitle{Background}
\label{background}

The phase evolution of a continuous gravitational wave signal arriving at a detector on
Earth can be written in the form:
\[
\phi(t) = \phi_0 + 2\pi \left(f_0\left[t+\tau(t)\right] + \frac{\dot{f}}{2}\left[t + \tau(t)\right]^2 \ldots \right),
\]
where $t$ is the time at the detector, and $\tau(t)$ is the time delay between the signal arrival
at the detector and the signal arrival at the solar system barycentre (SSB).

\end{frame}

\begin{frame}

\frametitle{Background}
\label{background}

The time delay term depends on the location of the source, and the position and velocity of the detector with
respect to the SSB, and comprises:
\[
\tau(t) = \Delta T_{\rm R}(t) + \Delta T_{\rm E} - \Delta T_{\rm S}, 
\]
where:

\begin{itemize}
\item $\Delta T_{\rm R}(t)$ is the geometric Roemer delay

\item $\Delta T_{\rm E}$ is the relativistic Einstein delay

\item $\Delta T_{\rm S}$ is the Shapiro delay

\end{itemize}

\end{frame}

\begin{frame}

\frametitle{Background}
\label{background}

For long duration signals the phase needs to be tracked precisely, so this time delay needs to be known
to $\mathcal{O}(\mu{\rm s})$ accuracy. For an observation time of a year, and a signal at 100 Hz, sky locations
separated by $15 \mu$ rad (near the ecliptic) produce 10\% mismatches between signals, so long wide-sky-area
searches require the SSB calculation to be repeated for many sky positions.

The code for the SSB calculated, see e.g.  \href{http://software.ligo.org/docs/lalsuite/lalpulsar/\_l\_a\_l\_barycenter\_8c\_source.html\#l00078}{\texttt{XLALBarycenter()}}, 
consists of many calls to \texttt{sin} and \texttt{cos}, so if needed many times could be a computational bottleneck. Is there a way to speed
up the calculation, but maintain accuracy?

\end{frame}

\begin{frame}

\frametitle{Reduced Order Modelling}
\label{reducedordermodelling}

Reduced order modelling (ROM) is basically a compression technique (similar to, e.g., Principal
Component Analysis).

\begin{itemize}
\item generate a ``training set'' of signal models (randomly) over a required parameter space

\item use modified  \href{https://en.wikipedia.org/wiki/Gram\%E2\%80\%93Schmidt\_process}{Gram-Schmidt process}  to form a
 minimal set of orthonormal bases from the ``training set'' that satisfy some constraint:

\begin{itemize}
\item give a small projection error of the \emph{current} bases onto the remaining training data

\item give a small \textbf{resdiual} when generating an interpolant from the current bases and
 comparing the interpolated models to the training set models

\end{itemize}

\end{itemize}

\end{frame}

\begin{frame}

\frametitle{Reduced Order Modelling}
\label{reducedordermodelling}

See, e.g., Appendix A \& B of \href{http://ukads.nottingham.ac.uk/abs/2013PhRvD..87l4005C}{Canizares \emph{et al}, PRD, 124005 (2013)}\footnote{\href{http://ukads.nottingham.ac.uk/abs/2013PhRvD..87l4005C}{http:/\slash ukads.nottingham.ac.uk\slash abs\slash 2013PhRvD..87l4005C}} for algorithms, and \texttt{greedycpp}\footnote{ \url{https://bitbucket.org/sfield83/greedycpp} }
code for

\end{frame}

\begin{frame}

\frametitle{Usage}
\label{usage}

Although potentially not relevant for current all-sky searches, this could be useful for:

\begin{itemize}
\item \textbf{long coherent follow-ups} of candidates with broad sky error regions

\item searches that \textbf{stochasticly sample the sky} (e.g. using an MCMC) at many points

\end{itemize}

Reduced order modelling (and the related \emph{Reduced Order Quadrature} for likelihood calculations)
is already implemented (but, yet to be written-up!) in the standard Bayesian pipeline known pulsar search
for narrow parameter searches including frequency and binary parameters.

\end{frame}

\begin{frame}

\frametitle{Future}
\label{future}

Even if this has limited applicability in current CW searches it may be relevant in other cases:

\begin{itemize}
\item CBC signals may need days-long templates in 3G detector era that require referencing to SSB

\item Many signals in LISA require referencing to the SSB

\item PTAs currently have quite sparse time samples, but in the future (e.g. SKA) there may be larger
 numbers of samples

\end{itemize}

\end{frame}

\mode<all>
\input{mmd-beamer-footer}

\end{document}\mode*

